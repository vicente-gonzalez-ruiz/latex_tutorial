\documentclass{article}

\usepackage{url}     % http://www.ctan.org/tex-archive/macros/latex/contrib/url
\usepackage{xparse}  % http://www.ctan.org/pkg/xparse

\begin{document}

1. \url{http://www.tug.org}\par
2. \url{www.tug.org}

%                        Command definition
%                           <----------->
\newcommand{\site}[2][http]{\url{#1://#2}}
%             ^    ^   ^         ^    ^
%             |    |   |         |    Second argument.
%             |    |   |         First argument.
%             |    |   Default value for the fist argument.
%             |    Number of arguments.
%             Command name.

3. \site{www.tug.org}\par % Only the second argument has been provided
4. \site[ftp]{ctan.org}   % Both arguments have been provided

\renewcommand{\site}[2][http]{[\url{#1://#2}]} % Includes "[" ... "]"

5. \site{www.tug.org}\par
6. \site[ftp]{ctan.org}

% Using \newcommand (or \renewcommand) only the first argument can be optional.
% To overcome this drawback, use:
\DeclareDocumentCommand{\site}{O{http://}O{www.ual.es}}{\{\url{#1#2}\}}

7. \site[][www.tug.org]\par     % We specify the second argument
8. \site[ftp://][ctan.org]\par  % We specify both arguments
9. \site                        % We don't specify arguments

% Definitions are easier than if you don't touch the arguments.
\renewcommand{\site}{\url}
10. \site{http://www.ual.es}

% Delete definition.
\let\site\relax

\end{document}
